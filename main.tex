
%% bare_jrnl.tex
%% V1.3
%% 2007/01/11
%% by Michael Shell
%% see http://www.michaelshell.org/
%% for current contact information.
%%
%% This is a skeleton file demonstrating the use of IEEEtran.cls
%% (requires IEEEtran.cls version 1.7 or later) with an IEEE journal paper.
%%
%% Support sites:
%% http://www.michaelshell.org/tex/ieeetran/
%% http://www.ctan.org/tex-archive/macros/latex/contrib/IEEEtran/
%% and
%% http://www.ieee.org/



% *** Authors should verify (and, if needed, correct) their LaTeX system  ***
% *** with the testflow diagnostic prior to trusting their LaTeX platform ***
% *** with production work. IEEE's font choices can trigger bugs that do  ***
% *** not appear when using other class files.                            ***
% The testflow support page is at:
% http://www.michaelshell.org/tex/testflow/


%%*************************************************************************
%% Legal Notice:
%% This code is offered as-is without any warranty either expressed or
%% implied; without even the implied warranty of MERCHANTABILITY or
%% FITNESS FOR A PARTICULAR PURPOSE!
%% User assumes all risk.
%% In no event shall IEEE or any contributor to this code be liable for
%% any damages or losses, including, but not limited to, incidental,
%% consequential, or any other damages, resulting from the use or misuse
%% of any information contained here.
%%
%% All comments are the opinions of their respective authors and are not
%% necessarily endorsed by the IEEE.
%%
%% This work is distributed under the LaTeX Project Public License (LPPL)
%% ( http://www.latex-project.org/ ) version 1.3, and may be freely used,
%% distributed and modified. A copy of the LPPL, version 1.3, is included
%% in the base LaTeX documentation of all distributions of LaTeX released
%% 2003/12/01 or later.
%% Retain all contribution notices and credits.
%% ** Modified files should be clearly indicated as such, including  **
%% ** renaming them and changing author support contact information. **
%%
%% File list of work: IEEEtran.cls, IEEEtran_HOWTO.pdf, bare_adv.tex,
%%                    bare_conf.tex, bare_jrnl.tex, bare_jrnl_compsoc.tex
%%*************************************************************************

% Note that the a4paper option is mainly intended so that authors in
% countries using A4 can easily print to A4 and see how their papers will
% look in print - the typesetting of the document will not typically be
% affected with changes in paper size (but the bottom and side margins will).
% Use the testflow package mentioned above to verify correct handling of
% both paper sizes by the user's LaTeX system.
%
% Also note that the "draftcls" or "draftclsnofoot", not "draft", option
% should be used if it is desired that the figures are to be displayed in
% draft mode.
%
\documentclass[journal]{IEEEtran}
\usepackage{blindtext}
\usepackage{graphicx}
\usepackage{booktabs}
\usepackage{subfigure}

% Some very useful LaTeX packages include:
% (uncomment the ones you want to load)


% *** MISC UTILITY PACKAGES ***
%
%\usepackage{ifpdf}
% Heiko Oberdiek's ifpdf.sty is very useful if you need conditional
% compilation based on whether the output is pdf or dvi.
% usage:
% \ifpdf
%   % pdf code
% \else
%   % dvi code
% \fi
% The latest version of ifpdf.sty can be obtained from:
% http://www.ctan.org/tex-archive/macros/latex/contrib/oberdiek/
% Also, note that IEEEtran.cls V1.7 and later provides a builtin
% \ifCLASSINFOpdf conditional that works the same way.
% When switching from latex to pdflatex and vice-versa, the compiler may
% have to be run twice to clear warning/error messages.






% *** CITATION PACKAGES ***
%
\usepackage{cite}
% cite.sty was written by Donald Arseneau
% V1.6 and later of IEEEtran pre-defines the format of the cite.sty package
% \cite{} output to follow that of IEEE. Loading the cite package will
% result in citation numbers being automatically sorted and properly
% "compressed/ranged". e.g., [1], [9], [2], [7], [5], [6] without using
% cite.sty will become [1], [2], [5]--[7], [9] using cite.sty. cite.sty's
% \cite will automatically add leading space, if needed. Use cite.sty's
% noadjust option (cite.sty V3.8 and later) if you want to turn this off.
% cite.sty is already installed on most LaTeX systems. Be sure and use
% version 4.0 (2003-05-27) and later if using hyperref.sty. cite.sty does
% not currently provide for hyperlinked citations.
% The latest version can be obtained at:
% http://www.ctan.org/tex-archive/macros/latex/contrib/cite/
% The documentation is contained in the cite.sty file itself.






% *** GRAPHICS RELATED PACKAGES ***
%
\ifCLASSINFOpdf
  % \usepackage[pdftex]{graphicx}
  % declare the path(s) where your graphic files are
  % \graphicspath{{../pdf/}{../jpeg/}}
  % and their extensions so you won't have to specify these with
  % every instance of \includegraphics
  % \DeclareGraphicsExtensions{.pdf,.jpeg,.png}
\else
  % or other class option (dvipsone, dvipdf, if not using dvips). graphicx
  % will default to the driver specified in the system graphics.cfg if no
  % driver is specified.
  % \usepackage[dvips]{graphicx}
  % declare the path(s) where your graphic files are
  % \graphicspath{{../eps/}}
  % and their extensions so you won't have to specify these with
  % every instance of \includegraphics
  % \DeclareGraphicsExtensions{.eps}
\fi
% graphicx was written by David Carlisle and Sebastian Rahtz. It is
% required if you want graphics, photos, etc. graphicx.sty is already
% installed on most LaTeX systems. The latest version and documentation can
% be obtained at:
% http://www.ctan.org/tex-archive/macros/latex/required/graphics/
% Another good source of documentation is "Using Imported Graphics in
% LaTeX2e" by Keith Reckdahl which can be found as epslatex.ps or
% epslatex.pdf at: http://www.ctan.org/tex-archive/info/
%
% latex, and pdflatex in dvi mode, support graphics in encapsulated
% postscript (.eps) format. pdflatex in pdf mode supports graphics
% in .pdf, .jpeg, .png and .mps (metapost) formats. Users should ensure
% that all non-photo figures use a vector format (.eps, .pdf, .mps) and
% not a bitmapped formats (.jpeg, .png). IEEE frowns on bitmapped formats
% which can result in "jaggedy"/blurry rendering of lines and letters as
% well as large increases in file sizes.
%
% You can find documentation about the pdfTeX application at:
% http://www.tug.org/applications/pdftex





% *** MATH PACKAGES ***
%
\usepackage[cmex10]{amsmath}
\usepackage{amssymb}
% A popular package from the American Mathematical Society that provides
% many useful and powerful commands for dealing with mathematics. If using
% it, be sure to load this package with the cmex10 option to ensure that
% only type 1 fonts will utilized at all point sizes. Without this option,
% it is possible that some math symbols, particularly those within
% footnotes, will be rendered in bitmap form which will result in a
% document that can not be IEEE Xplore compliant!
%
% Also, note that the amsmath package sets \interdisplaylinepenalty to 10000
% thus preventing page breaks from occurring within multiline equations. Use:
%\interdisplaylinepenalty=2500
% after loading amsmath to restore such page breaks as IEEEtran.cls normally
% does. amsmath.sty is already installed on most LaTeX systems. The latest
% version and documentation can be obtained at:
% http://www.ctan.org/tex-archive/macros/latex/required/amslatex/math/





% *** SPECIALIZED LIST PACKAGES ***
%
%\usepackage{algorithmic}
% algorithmic.sty was written by Peter Williams and Rogerio Brito.
% This package provides an algorithmic environment fo describing algorithms.
% You can use the algorithmic environment in-text or within a figure
% environment to provide for a floating algorithm. Do NOT use the algorithm
% floating environment provided by algorithm.sty (by the same authors) or
% algorithm2e.sty (by Christophe Fiorio) as IEEE does not use dedicated
% algorithm float types and packages that provide these will not provide
% correct IEEE style captions. The latest version and documentation of
% algorithmic.sty can be obtained at:
% http://www.ctan.org/tex-archive/macros/latex/contrib/algorithms/
% There is also a support site at:
% http://algorithms.berlios.de/index.html
% Also of interest may be the (relatively newer and more customizable)
% algorithmicx.sty package by Szasz Janos:
% http://www.ctan.org/tex-archive/macros/latex/contrib/algorithmicx/




% *** ALIGNMENT PACKAGES ***
%
%\usepackage{array}
% Frank Mittelbach's and David Carlisle's array.sty patches and improves
% the standard LaTeX2e array and tabular environments to provide better
% appearance and additional user controls. As the default LaTeX2e table
% generation code is lacking to the point of almost being broken with
% respect to the quality of the end results, all users are strongly
% advised to use an enhanced (at the very least that provided by array.sty)
% set of table tools. array.sty is already installed on most systems. The
% latest version and documentation can be obtained at:
% http://www.ctan.org/tex-archive/macros/latex/required/tools/


%\usepackage{mdwmath}
%\usepackage{mdwtab}
% Also highly recommended is Mark Wooding's extremely powerful MDW tools,
% especially mdwmath.sty and mdwtab.sty which are used to format equations
% and tables, respectively. The MDWtools set is already installed on most
% LaTeX systems. The lastest version and documentation is available at:
% http://www.ctan.org/tex-archive/macros/latex/contrib/mdwtools/


% IEEEtran contains the IEEEeqnarray family of commands that can be used to
% generate multiline equations as well as matrices, tables, etc., of high
% quality.


%\usepackage{eqparbox}
% Also of notable interest is Scott Pakin's eqparbox package for creating
% (automatically sized) equal width boxes - aka "natural width parboxes".
% Available at:
% http://www.ctan.org/tex-archive/macros/latex/contrib/eqparbox/





% *** SUBFIGURE PACKAGES ***
%\usepackage[tight,footnotesize]{subfigure}
% subfigure.sty was written by Steven Douglas Cochran. This package makes it
% easy to put subfigures in your figures. e.g., "Figure 1a and 1b". For IEEE
% work, it is a good idea to load it with the tight package option to reduce
% the amount of white space around the subfigures. subfigure.sty is already
% installed on most LaTeX systems. The latest version and documentation can
% be obtained at:
% http://www.ctan.org/tex-archive/obsolete/macros/latex/contrib/subfigure/
% subfigure.sty has been superceeded by subfig.sty.



%\usepackage[caption=false]{caption}
%\usepackage[font=footnotesize]{subfig}
% subfig.sty, also written by Steven Douglas Cochran, is the modern
% replacement for subfigure.sty. However, subfig.sty requires and
% automatically loads Axel Sommerfeldt's caption.sty which will override
% IEEEtran.cls handling of captions and this will result in nonIEEE style
% figure/table captions. To prevent this problem, be sure and preload
% caption.sty with its "caption=false" package option. This is will preserve
% IEEEtran.cls handing of captions. Version 1.3 (2005/06/28) and later
% (recommended due to many improvements over 1.2) of subfig.sty supports
% the caption=false option directly:
%\usepackage[caption=false,font=footnotesize]{subfig}
%
% The latest version and documentation can be obtained at:
% http://www.ctan.org/tex-archive/macros/latex/contrib/subfig/
% The latest version and documentation of caption.sty can be obtained at:
% http://www.ctan.org/tex-archive/macros/latex/contrib/caption/




% *** FLOAT PACKAGES ***
%
%\usepackage{fixltx2e}
% fixltx2e, the successor to the earlier fix2col.sty, was written by
% Frank Mittelbach and David Carlisle. This package corrects a few problems
% in the LaTeX2e kernel, the most notable of which is that in current
% LaTeX2e releases, the ordering of single and double column floats is not
% guaranteed to be preserved. Thus, an unpatched LaTeX2e can allow a
% single column figure to be placed prior to an earlier double column
% figure. The latest version and documentation can be found at:
% http://www.ctan.org/tex-archive/macros/latex/base/



%\usepackage{stfloats}
% stfloats.sty was written by Sigitas Tolusis. This package gives LaTeX2e
% the ability to do double column floats at the bottom of the page as well
% as the top. (e.g., "\begin{figure*}[!b]" is not normally possible in
% LaTeX2e). It also provides a command:
%\fnbelowfloat
% to enable the placement of footnotes below bottom floats (the standard
% LaTeX2e kernel puts them above bottom floats). This is an invasive package
% which rewrites many portions of the LaTeX2e float routines. It may not work
% with other packages that modify the LaTeX2e float routines. The latest
% version and documentation can be obtained at:
% http://www.ctan.org/tex-archive/macros/latex/contrib/sttools/
% Documentation is contained in the stfloats.sty comments as well as in the
% presfull.pdf file. Do not use the stfloats baselinefloat ability as IEEE
% does not allow \baselineskip to stretch. Authors submitting work to the
% IEEE should note that IEEE rarely uses double column equations and
% that authors should try to avoid such use. Do not be tempted to use the
% cuted.sty or midfloat.sty packages (also by Sigitas Tolusis) as IEEE does
% not format its papers in such ways.


%\ifCLASSOPTIONcaptionsoff
%  \usepackage[nomarkers]{endfloat}
% \let\MYoriglatexcaption\caption
% \renewcommand{\caption}[2][\relax]{\MYoriglatexcaption[#2]{#2}}
%\fi
% endfloat.sty was written by James Darrell McCauley and Jeff Goldberg.
% This package may be useful when used in conjunction with IEEEtran.cls'
% captionsoff option. Some IEEE journals/societies require that submissions
% have lists of figures/tables at the end of the paper and that
% figures/tables without any captions are placed on a page by themselves at
% the end of the document. If needed, the draftcls IEEEtran class option or
% \CLASSINPUTbaselinestretch interface can be used to increase the line
% spacing as well. Be sure and use the nomarkers option of endfloat to
% prevent endfloat from "marking" where the figures would have been placed
% in the text. The two hack lines of code above are a slight modification of
% that suggested by in the endfloat docs (section 8.3.1) to ensure that
% the full captions always appear in the list of figures/tables - even if
% the user used the short optional argument of \caption[]{}.
% IEEE papers do not typically make use of \caption[]'s optional argument,
% so this should not be an issue. A similar trick can be used to disable
% captions of packages such as subfig.sty that lack options to turn off
% the subcaptions:
% For subfig.sty:
% \let\MYorigsubfloat\subfloat
% \renewcommand{\subfloat}[2][\relax]{\MYorigsubfloat[]{#2}}
% For subfigure.sty:
% \let\MYorigsubfigure\subfigure
% \renewcommand{\subfigure}[2][\relax]{\MYorigsubfigure[]{#2}}
% However, the above trick will not work if both optional arguments of
% the \subfloat/subfig command are used. Furthermore, there needs to be a
% description of each subfigure *somewhere* and endfloat does not add
% subfigure captions to its list of figures. Thus, the best approach is to
% avoid the use of subfigure captions (many IEEE journals avoid them anyway)
% and instead reference/explain all the subfigures within the main caption.
% The latest version of endfloat.sty and its documentation can obtained at:
% http://www.ctan.org/tex-archive/macros/latex/contrib/endfloat/
%
% The IEEEtran \ifCLASSOPTIONcaptionsoff conditional can also be used
% later in the document, say, to conditionally put the References on a
% page by themselves.





% *** PDF, URL AND HYPERLINK PACKAGES ***
%
%\usepackage{url}
% url.sty was written by Donald Arseneau. It provides better support for
% handling and breaking URLs. url.sty is already installed on most LaTeX
% systems. The latest version can be obtained at:
% http://www.ctan.org/tex-archive/macros/latex/contrib/misc/
% Read the url.sty source comments for usage information. Basically,
% \url{my_url_here}.





% *** Do not adjust lengths that control margins, column widths, etc. ***
% *** Do not use packages that alter fonts (such as pslatex).         ***
% There should be no need to do such things with IEEEtran.cls V1.6 and later.
% (Unless specifically asked to do so by the journal or conference you plan
% to submit to, of course. )


% correct bad hyphenation here
\hyphenation{op-tical net-works semi-conduc-tor}


\begin{document}
%
% paper title
% can use linebreaks \\ within to get better formatting as desired
\title{Face Liveness Detection}
%
%
% author names and IEEE memberships
% note positions of commas and nonbreaking spaces ( ~ ) LaTeX will not break
% a structure at a ~ so this keeps an author's name from being broken across
% two lines.
% use \thanks{} to gain access to the first footnote area
% a separate \thanks must be used for each paragraph as LaTeX2e's \thanks
% was not built to handle multiple paragraphs
%

\author{Shiqi~Peng, Xiaoyang~Huo, Chen~Zuo, Bolin~Lai, Zhuoxun~He, Yue~Hu, Qiuying~Zhang, Qiaoyu~Lu, Ziyang~Zheng, Jinxiang~Liu, Qinwei~Xu, Chen~Ju% <-this % stops a space
\thanks{All the authors are with the Department
of Electronic Engineering, Shanghai Jiao Tong University.}}

% note the % following the last \IEEEmembership and also \thanks -
% these prevent an unwanted space from occurring between the last author name
% and the end of the author line. i.e., if you had this:
%
% \author{....lastname \thanks{...} \thanks{...} }
%                     ^------------^------------^----Do not want these spaces!
%
% a space would be appended to the last name and could cause every name on that
% line to be shifted left slightly. This is one of those "LaTeX things". For
% instance, "\textbf{A} \textbf{B}" will typeset as "A B" not "AB". To get
% "AB" then you have to do: "\textbf{A}\textbf{B}"
% \thanks is no different in this regard, so shield the last } of each \thanks
% that ends a line with a % and do not let a space in before the next \thanks.
% Spaces after \IEEEmembership other than the last one are OK (and needed) as
% you are supposed to have spaces between the names. For what it is worth,
% this is a minor point as most people would not even notice if the said evil
% space somehow managed to creep in.



% The paper headers
\markboth{Visual Computing Theory and Engineering Applications, Spring~2018}%
{Shell \MakeLowercase{\textit{et al.}}: Bare Demo of IEEEtran.cls for Journals}
% The only time the second header will appear is for the odd numbered pages
% after the title page when using the twoside option.
%
% *** Note that you probably will NOT want to include the author's ***
% *** name in the headers of peer review papers.                   ***
% You can use \ifCLASSOPTIONpeerreview for conditional compilation here if
% you desire.




% If you want to put a publisher's ID mark on the page you can do it like
% this:
%\IEEEpubid{0000--0000/00\$00.00~\copyright~2007 IEEE}
% Remember, if you use this you must call \IEEEpubidadjcol in the second
% column for its text to clear the IEEEpubid mark.



% use for special paper notices
%\IEEEspecialpapernotice{(Invited Paper)}




% make the title area
\maketitle


\begin{abstract}
%\boldmath
Face recognition is a widely used biometric approach. Face recognition technology has developed rapidly in recent years and it is more direct, user friendly and convenient compared to other methods. But face recognition systems are vulnerable to spoof attacks made by non-real faces. It is an easy way to spoof face recognition systems by facial pictures such as portrait photographs. A secure system needs Liveness detection in order to guard against such spoofing. In this work, face liveness detection approaches are categorized based on the various types techniques used for liveness detection. A review of the latest works regarding face liveness detection works is presented. Based on these works, some feasible improvements have been proposed in the further discussion.
\end{abstract}
% IEEEtran.cls defaults to using nonbold math in the Abstract.
% This preserves the distinction between vectors and scalars. However,
% if the journal you are submitting to favors bold math in the abstract,
% then you can use LaTeX's standard command \boldmath at the very start
% of the abstract to achieve this. Many IEEE journals frown on math
% in the abstract anyway.

% Note that keywords are not normally used for peerreview papers.
\begin{IEEEkeywords}
Face recognition, liveness detection, spoof attacks.
\end{IEEEkeywords}



% For peer review papers, you can put extra information on the cover
% page as needed:
% \ifCLASSOPTIONpeerreview
% \begin{center} \bfseries EDICS Category: 3-BBND \end{center}
% \fi
%
% For peerreview papers, this IEEEtran command inserts a page break and
% creates the second title. It will be ignored for other modes.
\IEEEpeerreviewmaketitle



\section{Introduction}

 \IEEEPARstart{T}{he} general public has a great need for security measures against spoof attacks. Biometrics is the fastest growing segment of such security industry. Some familiar techniques for identifacation are facial recognition, fingerprint recognition, handwriting verification, hand geometry, retina and iris scanners. Among these techniques, face recognition technology is more direct, user-friendly and convenient compared to other methods. Therefore, it has been applied to various security systems. However, in general, face recognition algorithms cannot distinguish between “live” faces and “non-live” faces, which is a major security issue. It is an easy way to spoof face recognition systems by facial pictures such as portrait photographs. To guard against such spoofing, a security system needs liveness detection.

The main task of a security system is the verification of an individual's identity. The primary reason for this is to prevent impostors from accessing from accessing protected resources. Traditional techniques for security purposes are passwords, but this technique can easily be lost or may be stolen. With the help of physical and biological properties of human beings, a biometric system can offer more security for a security system.

Liveness detection has been a very active research topic in fingerprint recognition and iris recognition communities in recent years. But in face recognition, approaches are very limited. The purpose of liveness detection is to differentiate the feature space into live and non-living. With the help of liveness detection, the performance of biometric system will improve. In face recognition, the usual attack methods may be classified into several categories. The classification is based on what verification proof is provided to face verification system, such as a stolen photo, recorded video, 3D face models, and so on. Anti-spoof problem should be well solved before face recognition systems widely applied in our daily life \cite{chakraborty2014overview}.

The remainder of this paper is organized as following. In section \uppercase\expandafter{\romannumeral2}, a review of the most interesting face liveness detection methods are presented. Then in section \uppercase\expandafter{\romannumeral3}, a discussion is presented citing the advantages and disadvantages of various face liveness detection approaches, and some feasible improvements are proposed. Finally, in section \uppercase\expandafter{\romannumeral4}, a conclusion is drawn.


% needed in second column of first page if using \IEEEpubid
%\IEEEpubidadjcol

% An example of a floating figure using the graphicx package.
% Note that \label must occur AFTER (or within) \caption.
% For figures, \caption should occur after the \includegraphics.
% Note that IEEEtran v1.7 and later has special internal code that
% is designed to preserve the operation of \label within \caption
% even when the captionsoff option is in effect. However, because
% of issues like this, it may be the safest practice to put all your
% \label just after \caption rather than within \caption{}.
%
% Reminder: the "draftcls" or "draftclsnofoot", not "draft", class
% option should be used if it is desired that the figures are to be
% displayed while in draft mode.
%
%\begin{figure}[!t]
%\centering
%\includegraphics[width=2.5in]{myfigure}
% where an .eps filename suffix will be assumed under latex,
% and a .pdf suffix will be assumed for pdflatex; or what has been declared
% via \DeclareGraphicsExtensions.
%\caption{Simulation Results}
%\label{fig_sim}
%\end{figure}

% Note that IEEE typically puts floats only at the top, even when this
% results in a large percentage of a column being occupied by floats.


% An example of a double column floating figure using two subfigures.
% (The subfig.sty package must be loaded for this to work.)
% The subfigure \label commands are set within each subfloat command, the
% \label for the overall figure must come after \caption.
% \hfil must be used as a separator to get equal spacing.
% The subfigure.sty package works much the same way, except \subfigure is
% used instead of \subfloat.
%
%\begin{figure*}[!t]
%\centerline{\subfloat[Case I]\includegraphics[width=2.5in]{subfigcase1}%
%\label{fig_first_case}}
%\hfil
%\subfloat[Case II]{\includegraphics[width=2.5in]{subfigcase2}%
%\label{fig_second_case}}}
%\caption{Simulation results}
%\label{fig_sim}
%\end{figure*}
%
% Note that often IEEE papers with subfigures do not employ subfigure
% captions (using the optional argument to \subfloat), but instead will
% reference/describe all of them (a), (b), etc., within the main caption.


% An example of a floating table. Note that, for IEEE style tables, the
% \caption command should come BEFORE the table. Table text will default to
% \footnotesize as IEEE normally uses this smaller font for tables.
% The \label must come after \caption as always.
%
%\begin{table}[!t]
%% increase table row spacing, adjust to taste
%\renewcommand{\arraystretch}{1.3}
% if using array.sty, it might be a good idea to tweak the value of
% \extrarowheight as needed to properly center the text within the cells
%\caption{An Example of a Table}
%\label{table_example}
%\centering
%% Some packages, such as MDW tools, offer better commands for making tables
%% than the plain LaTeX2e tabular which is used here.
%\begin{tabular}{|c||c|}
%\hline
%One & Two\\
%\hline
%Three & Four\\
%\hline
%\end{tabular}
%\end{table}


% Note that IEEE does not put floats in the very first column - or typically
% anywhere on the first page for that matter. Also, in-text middle ("here")
% positioning is not used. Most IEEE journals use top floats exclusively.
% Note that, LaTeX2e, unlike IEEE journals, places footnotes above bottom
% floats. This can be corrected via the \fnbelowfloat command of the
% stfloats package.

\section{Literature}

Many approaches are implemented in Face Liveness Detection. In this section, some of the most interesting liveness detection methods are presented.

\subsection{Frequency and Texture based analysis}

\textbf{(Zuo Chen and Lai Bolin)}

This approach is basic in Face Liveness Detection. The basic motivation is separating live face from fake face based on some details, such as the texture. Some works use spectrum of face. Live face and fake face may be different in frequency domain, or we can use the low frequency information and high frequency information separately to build a better classifier. In addition, there are several algorithms based on Local Binary Pattern (LBP) to analyse the textures on the given facial image \cite{chingovska2012effectiveness}\cite{maatta2011face}. We can understand this method in a more intuitive way: The live face is a 3D model and this means different parts reflect light differently. A photograph, however, is a platform. It reflects light like a mirror. The different ways of reflection cause the gap between live face and fake face in the texture or frequency. In the experiments, LBP performs well in capture of face texture.

The LBP operator \cite{ojala2002multiresolution} is regarded as a gray-scale invariant texture measure, derived from a general definition of texture in a local neighborhood. For each pixel, we compare its value with pixels arround it as shown in Fig. \ref{LBP}. The pixel larger than center is denoted as 1 and the one smaller than center is denoted as 0. Then we obtain a sequence of 1 and 0 clockwise or counterclockwise which describes texture information of an image. Finally, we convert the binary sequence to a decimal number. Fig. \ref{results_of_LBP} describes the results of LBP. Three raw images of one girl are under different illuminations. Feature maps obtained by LBP, however, seem similar. It's obvious that these feature maps capture the contour of eyes and nose。

\begin{figure}[!t]
\centering
\includegraphics[width=2.5in]{img/2-A-(3).png}
\caption{This is an example of LBP operator. Each pixel is given a value according to neighbor pixels. \cite{maatta2011face}}
\label{LBP}
\end{figure}

\begin{figure}[!t]
\centering
\includegraphics[width=2.5in]{img/2-A-(4).png}
\caption{Three raw images are different but the features obtained by LBP are similar to each other.}
\label{results_of_LBP}
\end{figure}

After computing LBP of given images, we can construct a classifier based on the feature map, such as neural network and SVM. The authors just use SVM to classify them in \cite{chingovska2012effectiveness}\cite{maatta2011face}. In order to get rid of dimensions of the feature, they compute histogram of each LBP feature map and use it as input to SVM. The structure is shown in Fig. \ref{lbp_based_approach}. The authors implement LBP on the whole image and on each block separately. These features are modeled by histograms and SVM.

\begin{figure*}[htbp]
\centering
\includegraphics[width=0.95\textwidth]{img/2-A-(1).png}
\caption{This is an approach \cite{chingovska2012effectiveness} used to implement Face Liveness Detection based on LBP. The authors use LBP and compute the histogram. SVM is used as classification. This work also gives another way that we can divide give images into several blocks and compute LBP on each block.}
\label{lbp_based_approach}
\end{figure*}

\bigskip

\textbf{(Experiment Results and Simple Analysis...)}


\subsection{Image Quality based analysis}

(Zhang Qiuying and Lu Qiaoyu)

Face spoof detection methods based on image quality analysis use image quality measures as features to determine whether a face is genuine or not.

In 2015, Di Wen et. al.\cite{wen2015face} proposed a face spoof detection method with image distortion analysis(IDA) to deal with 2D face mask attacks, such as printed photo and replayed video attacks, with real-time response(extracted from a single image with efficient computation). Instead of extracting features that capture the facial details, this method tries to capture the face image quality differences due to the different reflection properties of different materials, eg., facial skin, paper, and screen. And this method aims to improve the generalization ability under cross-database(different databases for training and testing) scenarios, which has seldom been explored in the biometrics community.

This method designed four discriminative features based on IDA that are capable of differentiating between genuine and spoof faces based on a single frame. Given a scenario where a genuine face or a spoof face (such as a printed photo or replayed video on a screen) is presented to a camera in the same imaging environment, the main difference between genuine and spoof face images is due to the “shape” and characteristics of the facial surface in front of the camera. According to the Dichromatic Reflection Model\cite{Shafer1992Using}, light reflectance $I$ of an object at a specific location $x$ can be composed into the following diffuse reflection $(I_d)$ and specular reflection $(I_s)$ components:
\begin{equation}
I(x)=I_d+I_s
\end{equation}

Since 2D spoof faces are recaptured from original genuine face images, so the distortion of a spoof faces consists of two parts: distortion in the diffuse reflection and distortion in the specular reflection, both of which are related to the spoofing medium. Besides the above distortions in the reflecting process, there is also distortion introduced by the capturing process. Although the capturing distortion can apply to both genuine and spoof faces. The spoof faces are more vulnerable to such distortion because they are usually captured in close distance to conceal the discontinuity of spoof medium frame.

Based on the above analysis, the major distortions in a spoof face image include: (1) specular reflection from the printed paper surface or LCD screen; (2) image blurriness due to camera defocus; (3) image chromaticity and contrast distortion due to imperfect color rendering of printer or LCD screen; and (4) color diversity distortion due to limited color resolution of printer or LCD screen.

\begin{figure*}[htbp]
\centering
\includegraphics[width=0.95\textwidth]{img/IDA_system.PNG}
\caption{The spoof detection algorithm based on Image Distortion Analysis.}
\label{lda_system}
\end{figure*}
\subsection{Optical Flow based analysis}

In light of differences in optical flow fields generated by movements of two-dimensional planes and three-dimensional objects, the paper \cite{Bao2009A} proposed a new liveness detection method for face recognition. Under the assumption that the test region is a two-dimensional plane, we can obtain a reference field from the actual optical flow field data. Then the degree of difference between the two fields can be used to distinguish between a three-dimensional face and a two-dimensional photograph.

Optical flow is the instantaneous speed of a moving spatial object's pixel movements on the projection plane. The research of optical flow is to use the timr-domain change and correlation of pixel intensity in image sequences to study the relationship between the intensity change by time and the object's structure and movement in the scene. The instantaneous change rate of intensity on specific point in projection plane is defiend as optical flow field, which contains information of instantaneous velocity vector of each pixel \cite{Barron1994Performance}.

The relative motion between a two-dimensional plane and observer can be divided into four basic types: translation, rotation, moving forward or backward and swing. The four motion types can generate different optical flow field, which is shown in Fig \ref{fig_C_1}.

\begin{figure}[!t]
\centering
\subfigure[translation]{
\label{Fig.sub.1}
\includegraphics[width=3in]{img/C_1_a}}
\subfigure[rotation]{
\label{Fig.sub.2}
\includegraphics[width=3in]{img/C_1_b}}
\subfigure[moving forward]{
\label{Fig.sub.3}
\includegraphics[width=3in]{img/C_1_c}}
\subfigure[swing]{
\label{Fig.sub.4}
\includegraphics[width=3in]{img/C_1_d}}
\caption{Optical flow fields generated by four basic types of relative motions}
\label{fig_C_1}
\end{figure}

The first three basic types of optical flow field generated by two-dimensional and three-dimensional objects are quite similar. The fourth type of optical flow field generated by two-dimensional and three-dimensional objects have more differences.

For the ideal case, the two-dimensional object's optical flow field can be presented as
\begin{equation}
\label{eq_C_1}
v_x = a_1x + b_1y + c_1
\end{equation}
\begin{equation}
\label{eq_C_2}
v_y = a_2x + b_2y + c_2
\end{equation}

We calculated the coefficients$a_1$,$b_1$,$c_1$,$a_2$,$b_2$,$c_2$, and emplpyed the optical flow field with these coefficients as the reference and compared with the test region optical flow field. We use D to represent the difference between the two optical flow fields:

\begin{equation}
\label{eq_C_3}
D = \frac{\sum\limits_{i=1}^m\sum\limits_{j=1}^n\sqrt{(a_1i + b_1j + c_1 - U_{ij})^{2} + (a_2i + b_2j + c_2 - V_{ij})^{2}}}{\sum\limits_{i=1}^m \sum\limits_{j=1}^n \sqrt{U_{ij}^{2} + V_{ij}^{2}}}
\end{equation}

The greater D is, the more likely it is a real face. Set a threshold T, when $D > T$ we conclude that it is a real face, otherwise it needs more detection.

The experiment was conducted on 3 groups of sample data. For the first group, they randomly translated, rotated and turned 100 face pictures in front of the camera. For the second group,  they folded and curled the pictures before shown to the camera, to make their surfaces not completely smooth. The third group was a real face experiment. As is shown in the Fig \ref{fig_C_2}, the greater T is, the higher the ratio of successful detection is.

By analyzing the optical flow field to detect real face, the liveness detection face recognition method proposed in the paper showed good performance in experiment. However, this method relies on the precise calculation of the optical flow field, so illumination change will have an impact on the result. In addition, this method is working under the assumption that the fake face is on a plane. It won't work on the three-dimensional face model or some seriously bended or folded face images. So in practice, it should be combined with other liveness detection methods to increase the rate of successful detection.

\begin{figure}[!t]
\centering
\includegraphics[width=3in]{img/C_2}
\caption{Experiment comparison}
\label{fig_C_2}
\end{figure}

\subsection{Blinking based analysis}

The eye-blinking based anti-spoofing technique was proposed by Lin Sun et. al. \cite{pan2007eyeblink} using Conditional Random Fields (CRFs). They develop a real-time liveness testing approach to resist photograph-spoofing in a non-intrusive manner for face recognition, which does not require any additional hardware except for a generic webcamera.

In general, a human can distinguish a live face or a photograph without much effort, since a prominent characteristic of live face is the occurrence of the non-rigid deformation and appearance change, such as mouth motion and expression variation. Accurate and relieable detection of these changes usually needs either high-quality input data or user collaboration. In the previous work, Kollreider et. al. \cite{kollreider2005evaluating} applied optical flow to the input video to obtain the information of face motion for live ness judgement. Frischholz et. al. \cite{frischholz2003avoiding} introduced an interactive approach requiring user to act an obvious response of head movement. With additional hardware, the vein map of the face by near infrared imaging, face thermogram \cite{socolinsky2003face} also could be applied in to liveness detection. But these methods have some drawbacks in common. They require high data quality, additional hardware, or high user collaboration, which is difficult for anti-spoofing.

Eyeblink is a physiological activity of rapid closing and opening of the eyelid. It is easy for the generic camera to capture two or more frames for each blink when the face looks into the camera. Hence, it is feasible to adopt eyeblink as a clue for anti-spoofing. The advantages of eyeblink based approach lie in: 1) it can complete in a non-intrusive manner, generally without user collaboration, 2) no extra hardware is required, 3) the eyeblink behavior is the prominently distinguishing character of a live face from a facial photo, which would be very helpful for liveness detection only from a generic camera.

The eyeblink behavior could be represented as a temporal image sequence after being digitally captured by the camera. Viola's cascaded Adaboost approach \cite{viola2001rapid} is a typical method to detect blink to classify each image in the sequence independently as one state of either closed eye or opened eye. But this method assumes all of the images in the temporal sequence are independent, missing the temporal information. To relax the independence assumption, an HMM (Hidden Markov Model) \cite{rabiner1989tutorial} models a sequence of observations by assuming that there is an underlying sequence of states drawn from a finite state set. Features of images can be regarded as the observations, and the eye state label is for the underlying states. HMM assumes that each state depends only on its immediate predecessor, and that each observation variable depends only on the current state, depicted in Fig. \ref{fig_D_1}. But for the task of eyeblink recognition, the two independence assumptions are too restrictive. Based on these facts, \cite{pan2007eyeblink} model eyeblink behaviors in an undirected Conditional Random Field framework, incorporated with a discriminative measure of eye states for simplifying the complex of inference and simultaneously improving the performance.

\begin{figure}[!t]
\centering
\includegraphics[width=1\linewidth]{img/D_1}
\caption{Graphical structure illustration of Hidden Markov Model.}
\label{fig_D_1}
\end{figure}

An eyeblink activity can be represented by an image sequence $\mathbb{S}$ consisting of $T$ images. Suppose that $\mathbb{S}$ is a random variable over observation sequences to be labeled, and $Y$ is a random variable over the corresponding label sequences to be predicted, components $y_i$ of $Y$ are assumed to range over a finite label set $\mathcal{Q}$. Let $G=(V,E)$ be a graph and $Y$ is indexed by the vertices of $G$. Then $(Y, \mathbb{S})$ is called a \textit{conditional random field (CRF)} \cite{lafferty2001conditional}.

Lin Sun et. al. \cite{pan2007eyeblink} yield a linear chain structure, shown in Fig. \ref{fig_D_2}. In this graphic model, observation window size $W$ is introduced to describe the conditional relationship between the current state and $(2W+1)$ temporal observations around the current one. It introduces the long-range dependencies in the model. Using the Hammersley and Clifford theorem \cite{li2009markov}, the joint distribution over the label sequence $Y$ given the observation $\mathbb{S}$ can be written as:

\begin{figure}[!t]
\centering
\includegraphics[width=0.95\linewidth]{img/D_2}
\caption{Graphical model of a linear-chain CRF, where the circles are variable nodes and the black boxes are factor nodes, in this example the state is conditioned on contexts of 3 neighboring observations, that is, $W=1$.}
\label{fig_D_2}
\end{figure}

\begin{equation}
\label{eq_D_1}
p_\theta(Y|\mathbb{S}) = \frac{1}{Z_\theta(\mathbb{S})} \exp (\sum_{t=1}^T \Psi_\theta(y_t,y_{t-1},\mathbb{S}))
\end{equation}
where $Z_\theta(\mathbb{S})$ is a normalized factor summing over all state sequences,
\begin{equation}
\label{eq_D_2}
Z_\theta(\mathbb{S}) = \sum_Y \exp (\sum_{t=1}^T \Psi_\theta(y_t,y_{t-1},\mathbb{S}))
\end{equation}

The potential function $\Psi_\theta(y_t,y_{t-1},\mathbb{S})$ is the sum of CRF features at time $t$:
\begin{equation}
\label{eq_D_3}
\Psi_\theta(y_t,y_{t-1},\mathbb{S}) = \sum_i \lambda_i f_i(y_t,y_{t-1},\mathbb{S}) + \sum_j \mu_j g_j(y_t,\mathbb{S})
\end{equation}

$f_i$ and $g_j$ are \textit{within-label} and \textit{between-observation-label} feature functions, respectively. $\lambda_i$ and $\mu_j$ are the feature weights associated with $f_i$ and $g_j$. $f_i$ and $g_j$ are defined as:
\begin{equation}
\label{eq_D_4}
f_i(y_t,y_{t-1},\mathbb{S}) = \mathbf{1}_{\{y_t=l\}} \mathbf{1}_{\{y_{t-1}=l'\}}
\end{equation}
\begin{equation}
\label{eq_D_5}
g_j(y_t,\mathbb{S}) = \mathbf{1}_{\{y_t=l\}} \mathcal{U}(I_{t-w})
\end{equation}
where $l, l' \in \mathcal{Q}, w\in[-W,W]$, $\mathcal{U}(\cdot)$ is the eye closity. Eye closity is a real-value discriminative feature, which is motivated by the idea of the adaptive boosting algorithm \cite{freund1997decision}, for the eye image measuring the degree of eye's closity. Defined as:
\begin{equation}
\label{eq_D_6}
\mathcal{U}_M(I) = \sum_{i=1}^M (\log \frac{1}{\beta_1}) h_i(I) - \frac{1}{2} \sum_{i=1}^M \log \frac{1}{\beta_i}
\end{equation}

Parameter estimation of $\theta=\{\lambda_1,\dots,\lambda_A,\mu_1,\dots,\mu_B\}$ is typically performed by MLE. Given a labeled training set $\{Y^{(i)}, \mathbb{S}^{(i)}\}_{i=1,\dots,N}$, the conditional log likelihood is:
\begin{equation}
\label{eq_D_7}
\begin{aligned}
L_\theta &= \sum_{i=1}^N \log (p_\theta(Y^{(i)}|\mathbb{S}^{(i)})) \\
&= \sum_{i=1}^N (\sum_{t=1}^T \Psi_\theta(y_t^{(i)},y_{t-1}^{(i)},\mathbb{S}^{(i)}) - \log Z_\theta(\mathbb{S}^{(i)}))
\end{aligned}
\end{equation}
Because $L_\theta$ is concave, every local optimum is also a global optimum. Finally, the optimization problem is solved by a limited-memory version of BFGS \cite{sha2003shallow}, a kind of quasi-Newton methods.

The inference tasks to label an unknown instance $Y^* = \arg \max_Y p(Y|\mathbb{S})$ can be performed efficiently and exactly by variants of the standard dynamic programming methods for HMM \cite{rabiner1989tutorial}.

To evaluate this approach, Lin Sun et. al. \cite{pan2007eyeblink} built a publicly available blinking video database. Using this blinking database, the CRF-based blinking detecction is compared with cascaded Adaboost and HMM approaches. The detection rate is shown in Tab. \ref{tab_D_1} and Tab. \ref{tab_D_2}.

\begin{table}[!htbp]
\centering
\caption{One-eye detection rate compared with the cascaded Adaboost and HMM.}
\label{tab_D_1}
\begin{tabular}{ccccc}
\toprule
\textbf{Data} & Cascaded AdaBoost & HMM &  CRF(W=2) \\
\midrule
\textbf{Frontal w/o glasses} & 96.5\% & 69.6\% & 93.8\% \\
\textbf{Frontal w/ thin rim glasses} & 60.0\% & 43.9\% & 85.6\% \\
\textbf{Frontal w/ black frame glasses} & 46.9\% & 42.5\% & 82.1\% \\
\textbf{Upward w/o glasses} & 96.5\% & 45.5\% & 82.6\% \\
\midrule
\textbf{Average} & 64.0\% & 49.6\% & 86.9\% \\
\bottomrule
\end{tabular}
\end{table}

\begin{table}[!htbp]
\centering
\caption{Two-eye detection rate compared with the cascaded Adaboost and HMM.}
\label{tab_D_2}
\begin{tabular}{ccccc}
\toprule
\textbf{Data} & Cascaded AdaBoost & HMM &  CRF(W=2) \\
\midrule
\textbf{Frontal w/o glasses} & 98.2\% & 80.4\% & 98.2\% \\
\textbf{Frontal w/ thin rim glasses} & 80.0\% & 60.6\% & 93.9\% \\
\textbf{Frontal w/ black frame glasses} & 71.9\% & 55.2\% & 89.6\% \\
\textbf{Upward w/o glasses} & 62.3\% & 59.1\% & 92.4\% \\
\midrule
\textbf{Average} & 78.1\% & 63.4\% & 93.7\% \\
\bottomrule
\end{tabular}
\end{table}

The comparison results show that the CRF approach outperforms the others. However, blinking-based liveness detection has some limitations. It would be affected by strong glasses reflection, which may cover eyes partially or totally.

\subsection{Video based analysis}

(Zheng Ziyang) The subtitle could be polished...

\subsection{3D Face Shape based analysis}

(Hu Yue and He Zhuoxun)

Most of the previous work focus on solving the attack of 2D fake masks, however, with development of 3D reconstruction and printing technologies, the primary assumption can no longer be maintained. Nesli Erdogmus and Sebastien Marcel extended their previous work \cite{erdogmus2013spoofing} on 2D mask attacks into 3D. In \cite{erdogmus2014spoofing} they provided a baseline study on the reported papers on the vulnerabilities to 3D mask attacks that is open-source and available for the research community to reuse. Besides, they provided new baselines which had better performance than original state-of-art baseline on 2D, 2.5D and 3D scenarios. And reporting comparative experimental results on two databases which will act as the missing link between the previous studies that have been done on Morpho database and the future studies that will be based on 3D Mask Attack Database (3DMAD) which is the first public spoofing database with facial masks.

In Fig. \ref{fig_3D_1}-a, there are some examples of grayscale texture (2D), depth map (2.5D) and 3D model format from left to right. And the top row is the real user while the bottom ones are a attacker wearing masks. In Fig. \ref{fig_3D_1}-(b-c), there are some real masks from websites.

They conducted two types of experiments. One is the face verification where the success rates of spoofing attacks with 3D masks are assessed using baseline face recognition algorithms. The results revealed their vulnerabilities to 3D facial mask attacks. The other one is anti-spoofing experiments in which mask attack/real face classification accuracy of aforementioned counter measure methods are measured. The parallel evaluations of Local Binary Pattern (LBP) \cite{kose2013countermeasure} based anti-spoofing methods on Morpho and 3DMAD databases made it possible to associate previously published results on the Morpho database with their current work and with possible future studies on 3DMAD. But their work considered different types of attacks separately, it is not what happened in reality. So constructing a robust system which can deal with different spoofing attacks scenario may be worth researching in the future.


\begin{figure}[!t]
\centering
\includegraphics[width=1\linewidth]{img/3D_1}
\caption{(a) The top row shows a real access from a user in grayscale texture (2D), depth map (2.5D) and 3D model format while an attacker wearing the same user¡¯s mask is displayed in the bottom (b-c) Facial masks obtained from ThatsMyFace.com}
\label{fig_3D_1}
\end{figure}

\section{Discussion}

Here, liveness detection approaches are categorized based on the type of liveness indicator used to assist the liveness detection of faces. Three main types of indicators were mainly used: motion, texture and life sign.



\section{Conclusion}

This work provided an overview of different approaches of face liveness detection.





% if have a single appendix:
%\appendix[Proof of the Zonklar Equations]
% or
%\appendix  % for no appendix heading
% do not use \section anymore after \appendix, only \section*
% is possibly needed

% use appendices with more than one appendix
% then use \section to start each appendix
% you must declare a \section before using any
% \subsection or using \label (\appendices by itself
% starts a section numbered zero.)
%


\appendices
\section{Proof of the First Zonklar Equation}
Some text for the appendix.

% use section* for acknowledgement
\section*{Acknowledgment}


The authors would like to thank...


% Can use something like this to put references on a page
% by themselves when using endfloat and the captionsoff option.
\ifCLASSOPTIONcaptionsoff
  \newpage
\fi



% trigger a \newpage just before the given reference
% number - used to balance the columns on the last page
% adjust value as needed - may need to be readjusted if
% the document is modified later
%\IEEEtriggeratref{8}
% The "triggered" command can be changed if desired:
%\IEEEtriggercmd{\enlargethispage{-5in}}

% references section

% can use a bibliography generated by BibTeX as a .bbl file
% BibTeX documentation can be easily obtained at:
% http://www.ctan.org/tex-archive/biblio/bibtex/contrib/doc/
% The IEEEtran BibTeX style support page is at:
% http://www.michaelshell.org/tex/ieeetran/bibtex/
\bibliographystyle{IEEEtran}
% argument is your BibTeX string definitions and bibliography database(s)
\bibliography{IEEEabrv,ref}
%
% <OR> manually copy in the resultant .bbl file
% set second argument of \begin to the number of references
% (used to reserve space for the reference number labels box)
%\begin{thebibliography}{1}

%\bibitem{IEEEhowto:kopka}
%H.~Kopka and P.~W. Daly, \emph{A Guide to \LaTeX}, 3rd~ed.\hskip 1em plus
%  0.5em minus 0.4em\relax Harlow, England: Addison-Wesley, 1999.

%\end{thebibliography}

% biography section
%
% If you have an EPS/PDF photo (graphicx package needed) extra braces are
% needed around the contents of the optional argument to biography to prevent
% the LaTeX parser from getting confused when it sees the complicated
% \includegraphics command within an optional argument. (You could create
% your own custom macro containing the \includegraphics command to make things
% simpler here.)
%\begin{biography}[{\includegraphics[width=1in,height=1.25in,clip,keepaspectratio]{mshell}}]{Michael Shell}
% or if you just want to reserve a space for a photo:


% You can push biographies down or up by placing
% a \vfill before or after them. The appropriate
% use of \vfill depends on what kind of text is
% on the last page and whether or not the columns
% are being equalized.

%\vfill

% Can be used to pull up biographies so that the bottom of the last one
% is flush with the other column.
%\enlargethispage{-5in}



% that's all folks
\end{document}


